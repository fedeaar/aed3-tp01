Una estrategia \textit{golosa} ---o \textit{greedy}--- es una estrategia de resolución de problemas, por lo general de optimización, que aprovecha las propiedades intrínsecas a una solución óptima para lograr resolver un problema de manera correcta y computacionalmente eficiente. En relación a la estrategia de \textit{programación dinámica}, el método se basa\footnote{ Ver Thomas H. Cormen; Charles E. Leiserson; Ronald L. Rivest y Clifford Stein. Introduction to algorithms.
2009. Sección 16.2: \textit{Elements of a greedy strategy}.} en la aplicación de la propiedad de \textit{subestructura óptima}\footnote{ No todo problema tiene este propiedad y no todo problema que la posee tiene una solución \textit{greedy}.}: una solución óptima a un problema incorpora soluciones óptimas a subproblemas relacionados. Sin embargo, a diferencia de esta estrategia, y del método más general de \textit{backtracking}, los algoritmos \textit{greedy} obtienen una solución por medio de decisiones que ``\textit{contemplan la información inmediatamente disponible, sin preocuparse por los efectos que estas decisiones puedan tener en el futuro}''\footnote{ Ver Gilles Brassard y Paul Bratley. Fundamentals of Algorithmics. 1995. Capítulo 6: \textit{Greedy Algorithms}.}. Un comportamiento que requiere que el problema tenga la propiedad de \textit{selección golosa}: una solución globalmente óptima se puede obtener a partir de decisiones localmente óptimas. 

El siguiente informe evalúa la aplicación del método sobre el problema de la \textit{selección de actividades}, que se enmarca dentro del más general \textit{problema de la fiesta}. Además, evalúa la eficiencia del algoritmo resultante de manera empírica.

$\\$
\noindent Palabras clave: \textit{Algoritmos golosos, estrategias algorítmicas, selección de actividades.}
